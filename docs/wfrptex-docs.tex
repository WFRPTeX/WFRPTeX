\documentclass[10pt,a4paper]{article}

\usepackage[T1]{fontenc}
\usepackage{libertine}
\usepackage{inconsolata}

\usepackage[top=1.5in,bottom=1in,left=1in,right=1in]{geometry}

\title{WFRP\TeX{} Documentation, Version 1.0.0}
\author{Ian Knight}
\date{}

\begin{document}
\maketitle

\section{Introduction}
WFRP\TeX{} is a \LaTeX{} package that comes with two new document classes
designed to make it easier to create WFRP content in \LaTeX{}. This document
will describe how to use the key features of the package to create your
own adventures and source material.

\section{Document classes}
There are two document classes provided by WFRP\TeX{}, for different kinds
of documents. At present, neither class takes any options, so just name the
class like normal. Both classes use exactly the same commands, so just pick
the one that fits your use-case the best.

\paragraph{\texttt{wfrp-long}} The first class is for longer documents that
are divided into chapters, like the Core Rulebook.

\paragraph{\texttt{wfrp-short}} The second class is for shorter documents
that do not have separate chapters, such as shorter adventure supplements.

\end{document}
